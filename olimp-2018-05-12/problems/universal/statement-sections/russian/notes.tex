В первом примере задана функция $x \downarrow y$ (ИЛИ-НЕ, Стрелка Пирса). Она удовлетворяет условию задачи. Например, отрицание из неё выражается так: $\neg x = x \downarrow x$, конъюнкция~---~$x \land y = (x \downarrow x) \downarrow (y \downarrow y)$, дизъюнкция~---~$x \lor y = (x \downarrow y) \downarrow (x \downarrow y)$

Во втором примере задана функция $x \land y$ (И, конъюнкция). Она не удовлетворяет условию задачи. Например, с её помощью нельзя выразить отрицание.