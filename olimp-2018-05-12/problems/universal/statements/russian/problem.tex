\begin{problem}{Функциональная полнота}{стандартный ввод}{стандартный вывод}{1 секунда}{256 мегабайт}

Олегу на день рождения подарили булеву функцию $f(x, y)$. Ему стало интересно, можно ли только с помощью этой булевой функции выразить любую другую булеву функцию. То есть, можно ли для каждой булевой функции $g(x_1, x_2, \ldots, x_n)$ составить эквивалентное ей выражение с использованием только переменных $x_1, x_2, \ldots, x_n$ и вызовов функции $f$.

\InputFile
Во входном файле задана функция $f(x, y)$. Её описание состоит из 4 цифр, каждая из которых может быть \t{0} или \t{1}. Первая цифра равна $f(0, 0)$, вторая~---~$f(0, 1)$, третья~---~$f(1, 0)$ и четвёртая~---~$f(1, 1)$.

\OutputFile
Выведите \t{YES}, если с помощью $f(x, y)$ можно выразить любую булеву функцию, и \t{NO} иначе.

\Examples

\begin{example}
\exmp{1000
}{YES
}%
\exmp{0001
}{NO
}%
\end{example}

\Note
В первом примере задана функция $x \downarrow y$ (ИЛИ-НЕ, Стрелка Пирса). Она удовлетворяет условию задачи. Например, отрицание из неё выражается так: $\neg x = x \downarrow x$, конъюнкция~---~$x \land y = (x \downarrow x) \downarrow (y \downarrow y)$, дизъюнкция~---~$x \lor y = (x \downarrow y) \downarrow (x \downarrow y)$

Во втором примере задана функция $x \land y$ (И, конъюнкция). Она не удовлетворяет условию задачи. Например, с её помощью нельзя выразить отрицание.

\end{problem}

