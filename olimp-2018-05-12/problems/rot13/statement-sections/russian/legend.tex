Докажите, что вы умеете отлаживать. Ниже представлена программа, которая зашифровывает строку из строчных латинских букв. Вам предлагается написать программу, которая расшифровывает получившуюся последовательность.


\bigskip

\noindent\begin{tabular}{|p{0.4\textwidth}|p{0.3\textwidth}|p{0.3\textwidth}|}
\hline
\textbf{Pascal} & \textbf{C} & \textbf{Python 3} \\
\hline
\begin{lstlisting}
program cypher;
var s:string;
i:integer;
begin
readln(s);
for i:=1 to length(s) do
write(chr(97+(ord(s[i])-84)mod 26));
end.
\end{lstlisting}&

\begin{lstlisting}
#include <stdio.h>
#include <string.h>
int main() {
char c[256];
scanf("%s", &c);
for (char *i=c;*i;++i)
*i=97+(*i-84)%26;
puts(c);
return 0;
}
\end{lstlisting}&

\begin{lstlisting}
import codecs as c
a = input()
b = c.encode(a, 'rot_13')
print(b)
\end{lstlisting} \\
\hline
\end{tabular}
